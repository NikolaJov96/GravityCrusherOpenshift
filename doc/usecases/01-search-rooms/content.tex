\section{Увод}

\subsection{Резиме}
Дефинисање сценарија употребе при претрази постојећих соба којима је могуће прикључити се.

\subsection{Намена документа и циљне групе}
Документ ће користити сви чланови пројектног тима у развоју пројекта, а може се 
користити и при писању упутства за употребу.

\subsection{Референце}
\begin{enumerate}
	\item Опис пројектног задатка
	\item Упутство за писање спецификације сценарија случаја употребе
\end{enumerate}

\subsection{Отворена питања}
\begin{table}[h!]
\centering
	
	\begin{tabu}{ || X[l] | X[l] | X[l] | X[l] || }
	\hline
	\textbf{Верзија} & \textbf{Датум} & \textbf{Кратак опис} & \textbf{Аутор} \\
	\hline
	\hline
	& & &\\ 
	\hline
	& & &\\
	\hline
	& & &\\
	\hline
	& & &\\
	\hline
	\end{tabu}
	\caption{Преглед отворених питања}
	\label{table:2}
		
\end{table}



\section{Сценарио размене порука у соби}

\subsection{Кратак опис}
Овде је представљен сценарио случаја коришћења система за претрагу постојећих соба за игру.
Собе се приказују у листи, а свакој соби придружена су дугмад за приступ соби као играч или
као посматрач. Уколико нека опција није омогућена, дугме ће бити обележено као неактивно.
Сценарио ће имати јединствени ток, при чему увек може бити прекинут, уколико корисник изађе из
истранице за претрагу постојећих соба.

\subsection{Ток догађаја}
\begin{enumerate}
	\item Систем приказује листу постојећих соба којима корисник може да приступи на бар неки
	      начин
	\item Корисник проналази собу којој жели да приступи
	\item Корисник притиска на дугме са жељеном акцијом
	\item Систем потврђује да корисник може да приступи одабраној соби у жељеној улози
	      и шаље га на страницу за игру
\end{enumerate}

\subsection{Посебни захтеви}
Нема.

\subsection{Предуслови}
Неопходно је да постоји бар једна соба којој корисник може да приступи.

\subsection{Последице}
Корисник је преусмерен на страницу за игру.
