\section{Увод}

\subsection{Резиме}
Дефинисање сценарија употребe током учествовања у игри.

\subsection{Намена документа и циљне групе}
Документ ће користити сви чланови пројектног тима у развоју пројекта, а може се 
користити и при писању упутства за употребу.

\subsection{Референце}
\begin{enumerate}
	\item Опис пројектног задатка
	\item Упутство за писање спецификације сценарија случаја употребе
\end{enumerate}

\subsection{Отворена питања}
\begin{table}[h!]
\centering
\small
	
	\begin{tabu}{ || X[l] | X[l] | X[l] | X[l] || }
	\hline
	\textbf{Верзија} & \textbf{Датум} & \textbf{Кратак опис} & \textbf{Аутор} \\
	\hline
	\hline
	1.0 & 10.03.2018. & Још увек се разматра како ће сама игра тачно изгледати. Потребно
	 је завршити имплементацију прототоипа игре, како би он могао да буде анализиран пре
	 доношења даљих одлука о садржају игре. & Никола Јовановић \\
	\hline
	& & &\\
	\hline
	& & &\\
	\hline
	& & &\\
	\hline
	\end{tabu}
	\caption{Преглед отворених питања}
	\label{table:2}
		
\end{table}



\section{Сценарио Играње игре}

\subsection{Кратак опис}
У питању је функционалност учествовања у игри, која обухвата активности које корисцник може да предузме унутар игре, као и информације које се кориснику презентују.

\subsection{Ток догађаја}
У наставку биће приказан главни (успешни) сценарио, као и престали алтернативни токови. Сценарио ће имати јединствени ток, при чему увек може бити прекинут, уколико корисник изађе са странице за играње игре. Ток догађаја се понавља у круг све до завршетка или прекида игре.

\begin{enumerate}
    \item Корисник задаје једну или више акција помоћу контолних елемената на страници,
          тастатуре или миша
    \item Систем прихвата акције, уколико је корисник у улози играча
    \item Систем нализира задате акције, уколико је корисник у улози играча
    \item Систем ажурира стање игре и графички га приказује на екрану, свим корисницима
          у соби
    \item Повратак на корак 1
\end{enumerate}


\subsubsection{Игра има победника или је истекло дозвољено време за игру}
\begin{enumerate}[label=5.\arabic*]
	\item Систем исписује поруку да је игра завршена и приказује резултате играча на
	      екран
	\item Корисник може још да остане у соби како би размењивао поруке са осталим
	      играчима
	\item Корисник излази из собе и завршава ток догађаја
\end{enumerate}

\subsection{Посебни захтеви}
Нема.

\subsection{Предуслови}
Игра мора да буде започета и кориснику мора да буде дозвољен приступ соби од стране система.

\subsection{Последице}
Чување опште статистике игре, као и статистике играча који су били пријављени на систем током игре.
