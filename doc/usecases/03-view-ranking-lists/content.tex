\section{Увод}

\subsection{Резиме}
Дефинисање сценарија употребе при прегледу ранг-л\^{и}ста.

\subsection{Намена документа и циљне групе}
Документ ће користити сви чланови пројектног тима у развоју пројекта, а може се 
користити и при писању упутства за употребу.

\subsection{Референце}
\begin{enumerate}
	\item Опис пројектног задатка
	\item Упутство за писање спецификације сценарија случаја употребе
\end{enumerate}

\subsection{Отворена питања}
\begin{table}[h!]
\centering
	
	\begin{tabu}{ || X[l] | X[l] | X[l] | X[l] || }
	\hline
	\textbf{Верзија} & \textbf{Датум} & \textbf{Кратак опис} & \textbf{Аутор} \\
	\hline
	\hline
	1.0 & 10.03.2018. & Одабрати конкретне критеријуме за рангирање играча. &
	Филип Мандић\\
	\hline
	& & &\\
	\hline
	& & &\\
	\hline
	& & &\\
	\hline
	\end{tabu}
	\caption{Преглед отворених питања}
	\label{table:2}
		
\end{table}



\section{Сценарио преглед ранг-л\^{и}ста}

\subsection{Кратак опис}
Прегледање ранг листа омогућава свим посетиоцима сајта да прегледају рангирање
пријављених играча по неколико различитих критеријума. Пријављени играчи могу видети и
своје рангирање на ранг-листама.

\subsection{Ток догађаја}
Сценарио ће имати јединствени ток, при чему увек може бити прекинут, уколико корисник
изађе са странице за преглед ранг-л\^{и}ста.

\begin{enumerate}
	\item Кориснику се приказује првих десет играча на некој од ранг-л\^{и}ста
	\item Корисник може да одабере који од критеријума рангирања жели да примени 
	\item Корисник притиска дугме за учитавање
	\item Систем приказује ранг-листу сортирану по изабраном критеријуму
	\item Корисник може да се креће кроз ранг листу притиском на стрелице за прелазак на
	      следећих десет места на листи или уношењем места на листи које жели да види
	\item Систем у реалном времену ажурира приказивање захтеваних информација
\end{enumerate}

\subsubsection{Корисник је пријављен на свој налог}
\begin{enumerate}[label=1.\arabic*]
	\item Корисник се посебно назначава на ранг-листи
	\item Сценарио се наставља кораком 2
\end{enumerate}

\subsection{Посебни захтеви}
Нема.

\subsection{Предуслови}
Да би корснику било посебно обележено његово рангирање на некој ранг-листи, неопходно је
да буде пријављен на систем. 

\subsection{Последице}
Нема.
